\documentclass[12pt,letterpaper]{article}
\usepackage[utf8]{inputenc}
\usepackage[T1]{fontenc}
\usepackage{amsmath}
\usepackage{amsfonts}
\usepackage{amssymb}
\usepackage{graphicx}
\usepackage{natbib}
\usepackage{hyperref}
\pagenumbering{gobble}

\begin{document}

%%%%%%% Custom Title Page %%%%%%%%
\begin{center}
	Senior Project Proposal for \\ IDS 4723 Senior Project

	\vskip 10em
	Prepared by \\ Dewayne Cushman

	\vskip 10em
	Prepared for \\ Dr. Shirley Mixon \\ Professor

	\vskip 10em
	Submitted on \\ \today
\end{center}
\clearpage
\baselineskip=24pt
\pagenumbering{arabic}

For my IDS 4723 Senior Project, I have chosen to develop a new language learning tool for Chata Anumpa (Choctaw Language) to build on my existing skills from my career as a software developer and to integrate my coursework in Native America Studies and Language Revitalization. The Choctaw Nation of Oklahoma has worked to provide basic language vocabulary online and through a mobile web browser, but the existing information is more reference oriented than learning focused. My goal of the newly developed app is to provide a Choctaw vocabulary list in excess of 200 words, definitions, the English language translations, simple examples of common phrases, and flash card equivalents to evaluate learning. While I am focusing exclusively on the Choctaw language for this project, I will be building the application to be able to support any language that has an orthography that is supported by the Unicode standard.

This application is focused on being accessible to new and younger Choctaw community members and those who are interested in learning the language and are comfortable with modern computer technologies and approaches. The goal of the application is to provide a basic knowledge of vocabulary and grammar structure in order to build a foundation for additional learning within classrooms or other community supported language learning systems along with providing limited conversational phrases that might be able to spark conversations with others in the community.

The proposed project will build on the latest technologies and design cues to provide a modern and immersive application focused on the learning goals stated above. By implementing a Spaced Repetition algorithm, the most challenging words and phrases will be surfaced more often in order to build strong recall and lasting mental connections for long-term memory retention.

I strongly believe in the principles of Free and Open Source Software (FOSS) and sharing of ideas with anyone who wants to use them. I will be developing the application and supporting materials through my website (\url{http://techmanities.tv}) and GitHub pages (\url{https://github.com/techmanities/choctaw-flashcards}) and gathering feedback from those community members during the project development.

The project deliverables will include:
\begin{itemize}
	\item Application - Running natively on Windows/MacOS/Linux and Android Platforms. Possibly iOS if I can find the money to renew my developer account with the Apple App Store.
	\item User manual - Covering the installation, configuration, and basic usage of the application on all the supported platforms
	\item Web Site - Providing high level overview of the application's purpose, goals, and links to the application supporting files
	\item GitHub Page - Providing access to the application's source code, tests, documentation sources, and community conversations and issues through GitHub's built in features.
	\item YouTube overview - A short series of YouTube videos covering high level overviews of the application and installation steps. A simplified visual manual option.
	\item Reference page - Compiling all of the linguistic and technological information reviewed for the project.
\end{itemize}


I expect the project to require approximately 100 hours or more of dedicated work to complete all of the items above. I believe this will be manageable during the remaining time before the submission date and I have strong personal motivation in seeing the project completed and being successful.

The initial proposed time is as follows:
\begin{itemize}
	\item 2021-03-22 Mon - Development Tools and Project Infrastructure Configured
	\item 2021-03-29 Mon - Minimum Viable Product available along with scaffolding for all other assets. I will also be requesting user feedback and testers at this time.
	\item 2021-04-05 Mon - Iterative additions and refinement including updates to all assets incorporating any user feedback.
	\item 2021-04-12 Mon - Application functionally complete, documentation and video asset editing while incorporating any user feedback.
	\item 2021-04-19 Mon - Final touches and editing complete, incorporating any user feedback.
	\item 2021-04-26 Mon - Final Project Submission.
	\item 2021-2022 - Application support and enhancement.
\end{itemize}

By the end of this project I will have completed a professional designed, developed, and tested application running on multiple operating systems that is publicly available on the Internet free of charge along with all supporting assets and documentation. Additionally, I will support requests and bug fixes after the fact as this project might form the basis of an expanded application after graduation. Depending on hosting costs and support, the app could be monetized through ads, but I do not expect that will be a problem. Possible enhancements based on my research and user feedback could be the addition of a Chata Chatbot for basic conversational practice, or my long term goal of implementing conversational communications in Choctaw along the lines of Siri or Alexa. These additional ideas are far too complicated to even start with the limited time available for this project, they are simply provided for additional context and to illustrate future possibilities.

\end{document}